\documentclass{article}

\usepackage{amssymb, amsmath, amsthm, verbatim}

\begin{document}


\renewcommand{\a}{\textbf{a}}
\renewcommand{\b}{\textbf{b}}
\renewcommand{\d}{\textbf{d}}
\newcommand{\e}{\textbf{e}}

\large

\begin{center}
\textbf{Homework \# 7} \\  
\end{center}



\medskip


\newcommand{\normal}{\mathcal{N}}

Reading
\begin{itemize}
\item I've posted an excerpt from G. Lawler's book on stochastic processes relating to Poisson processes.   Take a look, it covers similar ideas as mentioned in the lecture.
\item Section $11.3$ of Bishop discusses Gibbs sampling.
\end{itemize}

\begin{enumerate} 



\item (This problem is related to exercise $4.1$ in chapter $2$ of the book Stochastic Simulation by Asmussen and Glynn).   The Gompertz-Makecham distribution is used as a lifetime distribution by life insurance companies.   To explain the distribution, let $X(t)$ be a Poisson process with rate $\lambda(t) = a + b\text{e}^{ct}$ where $a = 5 \cdot 10^{-4}$, $b=7.5858\cdot 10^{-5}$, $c = \log(1.09144)$.   If a person is, say, $45$ years old, then the first jump time of $X(t)$ after $t=45$ is used as a model for the person's age when they die.   
\begin{enumerate}
\item As a warm-up, write a function that samples the jump times of a Poisson process with constant rate $\lambda$ that occur prior to some specified time $t$.  
\item Write a sampler that generates the Poisson process $X(t)$ up to $t=100$.  Use the accept-reject algorithm that we talked about in class based on a constant rate Poisson process.   Show that the accept-reject algorithm is valid.  (Hint: to show the validity of the accept-reject algorithm show that it will sample a Poisson process with a jump in $[t, t+\Delta t]$ with probability approximately $\lambda(t) \Delta t$.  We did this in class.) 
\item Derive a formula for the cdf of the age upon death of a $60$ year old.   (Don't just quote the result I presented in class, derive it yourself.  Use the idea of splitting up time into small time intervals and taking a limit as those intervals go to $0$ in size.)
\item Compute the probability a $60$ year old lives to be $90$ in two ways
\begin{enumerate}
\item Use your languages integrate function to evaluate the formula for this probability based on your cdf in (c).
\item Use a Monte Carlo integration approach and your sampler from (b).
\end{enumerate}
\end{enumerate}


\item Consider a Poisson process with rate $\lambda(t) = 1/\sqrt{t}$.  Notice that $\lambda(t)$ is unbounded.
Calculate the pdf of the first jump time.

\item 
\begin{enumerate}
\item Redo problem $4b$ of homework $4$, but this time use a Gibbs sampler to sample from $Y$ rather than a Metropolis-Hastings sampler.
\item Explain why you cannot use a Gibbs sampler based on switching a single coordinate (i.e. die role) to sample die rolls for problem $2$ of homework $6$.
\item Construct a Gibbs sampler to sample die rolls for problem $2$ of homework $6$ by choosing two dice, computing their marginal distribution given all other die rolls, and sampling from the marginal  to produce the Markov chain update.  Such a sampler is not a Gibbs sampler in the sense that it considers one coordinate at a time, but it is a Gibbs sampler in the sense that it uses marginal distributions to update the chain.
\end{enumerate}

\end{enumerate}



\end{document}
